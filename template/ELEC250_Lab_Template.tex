%----------------------------------------------------------------------------
%----------------------------------------------------------------------------
%				    	SETUP
%----------------------------------------------------------------------------
%----------------------------------------------------------------------------

\documentclass[12pt]{article}

%----------------------------------------------------------------------------
%			  	   PACKAGES
%----------------------------------------------------------------------------

%% Fonts and Symbols
%% --------------------------
\usepackage[T1]{fontenc}			% better font encoding
\usepackage[utf8]{inputenc}		% better font encoding
%\usepackage[usenames,dvipsnames,svgnames,table]{xcolor}		% allow colour
\usepackage{amsmath,amssymb,amsthm,textcomp, xfrac}		% math symbols, etc
%\usepackage{color}
\usepackage[hidelinks]{hyperref}	% hyperlinks, see config in LAYOUT AND STYLING


%% Graphics
%% --------------------------
\usepackage{graphicx}			% allow insertion of images
\graphicspath{ {./graphics/} }		% the relative path to the graphics folder
%\usepackage{tikz}				% vector graphics
%\usepackage{pgfplots}			% plots in vector graphics


%% Tables
%% --------------------------
\usepackage{booktabs}			% better tables, discourages vertical rulings
%\usepackage{tabularx}
%\usepackage{enumerate}		
\usepackage{multicol}			%allow multi columns


%% Layout Alteration
%% --------------------------
\usepackage{lastpage}
\usepackage{fancyhdr}			% see config in LAYOUT AND STYLING
\usepackage{fullpage}			% set full page margins
%sideways figures
%\usepackage{rotating}
%\usepackage{pdflscape}
%\usepackage{parskip}			% disable indents

%% Units
%% --------------------------
%\usepackage{siunitx}
%\usepackage{cancel}
%\sisetup{load-configurations = abbreviations}
%\sisetup{per-mode = symbol}

%% Misc
%% --------------------------
%\usepackage{mhchem}			% chemistry


%----------------------------------------------------------------------------
%		     MACROS AND COMMANDS
%----------------------------------------------------------------------------

%type Y - even column width - centered
% must include tabularx package
%\newcolumntype{Y}{>{\centering\arraybackslash}X}	

% Defines a new command for the horizontal lines, change thickness here
\newcommand{\HRule}{\rule{\linewidth}{0.5mm}} 	

% ???
\newcommand{\linia}{\rule{\linewidth}{0.5pt}}

% scientific notation  use \e
\providecommand{\e}[1]{\ensuremath{\times 10^{#1}}}

% diferential
\def \d {\ensuremath{\mathrm{d}}}

%----------------------------------------------------------------------------
%		   	LAYOUT AND STYLING
%----------------------------------------------------------------------------

% custom footers and headers
% must include fancyhdr package
\pagestyle{fancy}
\lhead{}
\chead{}
\rhead{}
\lfoot{}
\cfoot{\thepage\ of \pageref{LastPage}}
\rfoot{}
\renewcommand{\headrulewidth}{0pt}
\renewcommand{\footrulewidth}{0pt}


%%section style
%\usepackage{titlesec}
%\titleformat{\section}[runin]
%{\normalfont\bfseries}
%{\thesection.}{.5em}{}[]
%
%\titleformat{\subsection}[runin]
%{\normalfont\bfseries}
%{\thesubsection}{.5em}{}[]
%\setcounter{secnumdepth}{0} %dont number sections

%\hypersetup{
%%    	colorlinks=false, 		% set true if you want colored links
%   		linktoc=all,     			% set to all if you want both sections and subsections linked
%%  		linkcolor=blue,  			% choose some color if you want links to stand out
%}


%----------------------------------------------------------------------------
%----------------------------------------------------------------------------
%				   DOCUMENT
%----------------------------------------------------------------------------
%----------------------------------------------------------------------------

\begin{document}

%----------------------------------------------------------------------------
%				    TITLE PAGE
%----------------------------------------------------------------------------

\begin{titlepage}

\center
 
% Header
\textsc{\LARGE University of Victoria}\\[1cm] 	% Name of your university/college
\textsc{\Large ELEC 250}\\[0.5cm] 			% Major heading such as course name
\textsc{\large Linear Circuits: 1}\\[0.5cm] 		% Minor heading such as course title


% Lab Title
\HRule \\[0.4cm]
{ \huge \bfseries Lab X - Title}\\[0.2cm] % Title of your document
\HRule \\[1.5cm]
 
 
%Lab Instructor Details
\begin{minipage}{0.7\textwidth}
\begin{flushleft} 

\large\emph{Instructor:} \\
Dr. Nikitas \textsc{Dimopoulos} \\
\vspace{12 pt}
\emph{Teaching Assistant:} \\
Zhen \textsc{Li}

\end{flushleft}
\end{minipage}
~
%% No content here, but it keeps the alignment of the instructor/TA
%% box correct.
%% Consider revising.
\begin{minipage}{0.1\textwidth}
\begin{flushright} \large
%Dr. Barbara \textsc{Sawicka} \\
\vspace{12 pt}
%\emph{Teaching Assistant:} \\
%Vahid \textsc{Moradi}
\end{flushright}
\end{minipage}\\[2cm]


% Lab members
\Large Clayton \textsc{Kihn}
\large V00794569	\\
\Large Yves \textsc{Senechal}
\large V00213837	\\
\Large Tyler \textsc{Stephen}
\large V00812021	\\
A01 - B01\\[1.5cm] 


% Date
{\large \today}\\ % Date, change the \today to a set date if you want to be precise

% Logo
\begin{figure}[b]	 % put logo at bottom of the page
	\centering
	\includegraphics[scale=0.3]{UVic_logo}
\end{figure}

\end{titlepage}

%----------------------------------------------------------------------------
%			  TABLE OF CONTENTS
%----------------------------------------------------------------------------

\tableofcontents
\pagebreak

%\listoffigures
%\pagebreak

%----------------------------------------------------------------------------
%				    BODY
%----------------------------------------------------------------------------

\section{Object}\label{sec:object}
The object of the experiment can be copied from the manual:
\begin{itemize}
	\item A list may be helpful
	\item It makes things look neater
	\item And requires less typing
\end{itemize}

\section{Results}\label{sec:results}
The results are basically an account of the experimental setup, methods of measurement and the results obtained. Simple circuit or block diagrams should be included in this section to explain the experimental procedure. The use of simple and clear diagrams is what distinguishes a good report from a poor one.

\section{Discussion and Conclusion}\label{sec:d_and_c}
The discussion and conclusion should answer the questions that are posed in the procedure section of the experiment. Any special observations made by the student can be recorded here.


%----------------------------------------------------------------------------
%				  LaTeX TIPS
%----------------------------------------------------------------------------

\pagebreak

\section{LaTeX Tips}\label{sec:tips}
Check the source file for additional information in the comments.
\subsection{Symbols}\label{sec:symbols}
Most math symbols and all equations are bounded by \$ delimiters. \verb|$ A=\pi r^{2} $| produces $ A=\pi r^{2} $. To find the appropriate symbol you will have to use a LaTeX IDE with a built in symbol editor or use another program to produce the code and copy-and-paste it into your document.

\subsection{Figures}\label{sec:figures}
\begin{verbatim}
\begin{figure}[h]
	\centering
	\includegraphics[width=0.75\textwidth]{Uvic_logo}
	\caption{A logo used by the Univesity of Victoria}
	\label{fig:uvic_logo}
\end{figure}
\end{verbatim}

\begin{figure}[h] 	% placement in document [htbp]: here, top, bottom, special page
	\centering		% centers the image
	\includegraphics[width=0.75\textwidth]{Uvic_logo} 	
		% will search root directory and anything listed
		% in \graphicspath for the image
		% it is common to specify [width=0.5\textwidth] as an agument
		% \textwidth can be used as a variable for the entire pagewidth.
		% 0.5\textwidth refers to 50% of the page.
	\caption{A logo used by the Univesity of Victoria}
	\label{fig:uvic_logo}
\end{figure}

A good tutorial for the use of figures can be found at: \url{http://en.wikibooks.org/wiki/LaTeX/Floats,_Figures_and_Captions}

\pagebreak
\subsection{Tables}\label{sec:tables}
\begin{verbatim}
\begin{table}[h]
	\centering
	\begin{tabular}{llr}
		\hline
		\multicolumn{2}{c}{Item} \\
		\cline{1-2}
			Animal   	& Description 	& Price (\$) \\
		\hline
			Gnat		& per gram	& 13.65      \\
				        & each       	& 0.01       \\
			Gnu		& stuffed     	& 92.50      \\
			Emu		& stuffed		& 33.33      \\
			Armadillo	& frozen		& 8.99       \\
		\hline
	\end{tabular}
	\caption{Exotic meat prices}
	\label{table:meats}
\end{table}
\end{verbatim}

\begin{table}[h]			% placement in document [htbp]: [h]ere, [t]op, [b]ottom, special [p]age
	\centering
	\begin{tabular}{llr}	% specifies the number of columns and their justification
					% columns can be [l]eft-justified, [c]entered, [r]ight-justified
					% the number of arguments after {tabular} corresponds to the number of columns
					% vertical lines can be added by placing | vertical bars in the argument
					% e.g. { | c c c | } has three centered columns with vertical lines at the ends
					% of the table
					% { | c | c | c | } has vertical lines separating all cells

		\hline		% a horizontal line
		
		\multicolumn{2}{c}{Item} \\		% {number of columns to span}{lcr}{title}
								% \\ indicates the end of a line
								
		\cline{1-2}		% \cline[ i - j } : line that spans columns i to j
		
			Animal   	& Description 	& Price (\$) \\	% & separates data between cells
											% && indicates a blank cell
											% \\ must end every row
		\hline
			Gnat		& per gram	& 13.65      \\
				        & each       	& 0.01       \\
			Gnu		& stuffed     	& 92.50      \\
			Emu		& stuffed		& 33.33      \\
			Armadillo	& frozen		& 8.99       \\
		\hline
	\end{tabular}
	\caption{Exotic meat prices}
	\label{table:meats}
\end{table}
\textit{Apparently} tables are more readable if the vertical rulings are omitted. I'm inclined to agree.\\
A good tutorial for the use of tables can be found at: \url{http://en.wikibooks.org/wiki/LaTeX/Tables}

\subsection{Labels and References}\label{sec:labels_and_refs}
LaTeX's dynamic referencing system gives it an advantage over other multi-user document tools. References point to assigned labels rather than a pre-defined numbering. Changing the order and number of references will leave the citations untouched if label referencing is used.


The \verb|\label{}| tag should be attached to all sections, figures and tables. To reference these elements, use the \verb|\ref{}| command. To reference the table in Section \ref{sec:tables}, you would write \verb|Table \ref{table:meats}| which will appear as Table \ref{table:meats}.


A consistent naming schema will make collaboration easier. Labels should be implemented with the corresponding prefix:
\begin{table}[h]
	\centering
	\begin{tabular}{ l l }
	Sections		& \verb|{sec:}|		\\
	Figures		& \verb|{fig:}|		\\
	Tables		& \verb|{table:}|		\\
	\end{tabular}
\end{table}

You may encounter a situation where a citation or page number appears as \verb|??|. This often occurs when major changes have occured to the reference or page order. The LaTeX compiler requires two executions of the typesetting function to correctly address the references: one to build the .aux file and another to read from it. The compiler is often nice enouch to pass a warning when the .aux file has undergone significant changes to its references and prompts you do another typesetting.

\subsection{Resources}\label{sec:resources}
\begin{itemize}
	\item \underline{\href{https://www.youtube.com/user/14mech14/videos}{Video playlist} } from McMaster that covers the installation and use of LaTeX. Uses TeXShop for examples. Covers document setup, tables, figures, bibliographies and some other stuff I haven't watched yet.
\end{itemize}

%----------------------------------------------------------------------------
%----------------------------------------------------------------------------
%			DO NOT DELETE BELOW
%----------------------------------------------------------------------------
%----------------------------------------------------------------------------

\end{document}